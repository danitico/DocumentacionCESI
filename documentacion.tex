%%%%%%%%%%%%%%%%%%%%%%%%%%%%%%%%%%%%%%%%%%%%%%%%%%%%%%%%%
%                                                       %
% La clase: 11pt o 10pt, es un draft o copia definitiva %
% Que tipo de codificacion usa, ...                     %
%                                                       %
%%%%%%%%%%%%%%%%%%%%%%%%%%%%%%%%%%%%%%%%%%%%%%%%%%%%%%%%%
\documentclass[11pt,twoside,a4paper]{book}
\usepackage{fancyhdr}
\usepackage{fancybox}
\usepackage[T1]{fontenc}
\usepackage[utf8]{inputenc}
\usepackage[spanish]{babel}
\usepackage{amsfonts}
\usepackage{latexsym}
\usepackage{graphicx}
\usepackage{floatflt}
\usepackage{epsfig}
\usepackage{subfigure}
\usepackage{mathrsfs}
\usepackage{amssymb}
\renewcommand{\baselinestretch}{1.2}
\title{\Huge Servidor Web Tomcat}
\author{José Manuel Cuevas Muñoz, Daniel Ranchal Parrado, Carlos Romeo Muñoz}
\date{9 de Diciembre 2018}
%%%%%%%%%%%%%%%%%%%%%%%%%%%
%                         %
% Comenzamos el documento %
%                         %
%%%%%%%%%%%%%%%%%%%%%%%%%%%
\begin{document}
\maketitle
%%%%%%%%%%%%%%%%%%%%%%%%%%%
%
% Algunos ajustes previos
%
%%%%%%%%%%%%%%%%%%%%%%%%%%%

\renewcommand\bibname{Bibliografía}
\renewcommand\tablename{Tabla}

\pagenumbering{arabic} \fancyhf{} \pagestyle{fancy}
\fancyhead[LO]{\rightmark} % En las pginas impares, parte izquierda del encabezado, aparecer el nombre de seccin
\fancyhead[RE]{\leftmark} % En las pginas pares, parte derecha del encabezado, aparecer el nombre de captulo
\fancyhead[RO,LE]{\thepage} % Nmeros de pgina en las esquinas de los encabezados
\setlength{\headheight}{14pt}
\renewcommand{\sectionmark}[1]{\markright{{\thesection. #1}}} % Formato para la seccin: N.M. Nombre
%%%
%con esto creamos el ndice
\tableofcontents
%
% Contenido TRABAJO
%

\chapter{Definición del Sistema}
\section{Objetivos y definición del sistema}
El objetivo que se pretende es hacer pruebas exhaustivas, es decir, hacer un benchmarking para calcular el rendimiento del servicio web Tomcat. Para hacer este tipo de pruebas, se parte de la siguiente información.
El sistema operativo bajo el que se están haciendo las pruebas es CentOS, cuyo servidor está alojado en Azure.
\newline
Para la instalación de Tomcat, se ha utilizado el servicio de virtualización docker para instalarlo de una manera sencilla. Para que el host, es decir, el servidor pueda dar el servicio que provee el contenedor de docker, hay que ligar el puerto del contenedor en el que está sirviendo Tomcat con el puerto que nosotros queramos en el host, aunque los autores de este guión y trabajo hemos elegido el 80, el puerto por excelencia para el servidor web.

\section{Servicios y sus posibles resultados}
El servidor web Tomcat nos provee una gran cantidad de utilidades para aprovechar al máximo este mismo. La primera y la más importante es el despliegue de una página web. Tomcat nos permite hacerlo de manera estática, es decir, configurar la aplicación antes de activar este servicio o de manera dinámica, que es lo más utilizado para servidores que están en producción.
\newline
Para poder realizar esto de una manera sencilla, Tomcat nos da una herramienta llamada "Manager", que cómo se ha comentado antes, se pueden hacer despliegues y eliminación (undeploy) de cualquier aplicación web además de darnos una lista de las aplicaciones que ya están desplegadas.
\newline
Pero las utilidades de "Manager" no son sólo esas, este gestor nos da la opción de poder ver las estadísticas de las sesiones de cualquier aplicación y el estado del servidor. Otra de las funcionalidades que tiene, al igual que el servidor httpd y apache2, es la posibilidad de trabajar con "Virtual Host".
\newline
Otra característica interesante que nos ofrece Tomcat es la gestión de los usuario y los roles en las distintas aplicaciones web. Esto evita que tengamos una tabla para cada aplicación web que se tenga. En temas de seguridad, Tomcat nos permite la configuración de los certificados SSL/TLS para que nuestra página tenga el protocolo https.
\newline
Para este experimento, se distinguirán los siguientes resultados:

\begin{itemize}
  \item \textbf{La carga correcta de las imágenes en un tiempo considerablemente bueno.}
  \item \textbf{La carga correcta de las imágenes en un tiempo pésimo.}
  \item \textbf{La carga parcial de las imágenes.}
  \item \textbf{Fallo del servidor web. Ninguna imágen es servida.}
\end{itemize}


\section{Métricas}
Para poder medir la eficacia y la actuación del servidor web Tomcat se han considerado los siguientes parámetros a examinar:

\begin{itemize}
  \item \textbf{Peticiones por segundo}
  \item \textbf{Tiempo por cada grupo de peticiones al mismo tiempo}
  \item \textbf{Tiempo por cada petición}
  \item \textbf{Uso de la CPU}
  \item \textbf{E/S Disco}
  \item \textbf{E/S Red}
\end{itemize}

\pagebreak
\section{Parámetros}
Los parámetros que pueden afectar a la medición del rendimiento del servidor web Tomcat son los siguientes:

\begin{itemize}
  \item \textbf{La lejanía con el centro de datos}
  \item \textbf{Los procesos que se ejecutan en segundo plano}
  \item \textbf{Estado de nuestra conexión a Internet}
  \item \textbf{Hardware empleado en el servidor}
\end{itemize}

Para poder mitigar los efectos que producen estos parámetros,la prueba se ejecutará varias veces y se harán los cáculos oportunos para representar de una manera correcta las estadísticas.

\chapter{Evaluacin del Sistema}
Definicin de la experimentacin a realizar. Qu evaluacin se seguir, cual ser la carga de trabajo, qu experimentos se van a realizar y finalmente se analizarn los resultados de la experimentacin.

\section{Tcnicas de evaluacin}
Qu tcnica de evaluacin se seguir.

\section{Carga de trabajo}
Cual ser la carga de trabajo, en qu consiste.

\section{Diseo de Experimentos}
Cmo se han diseado los experimentos. En qu consiste la experimentacin a realizar.

\section{Anlisis de los resultados}
Estudio y anlisis de los resultados provenientes de la experimentacin. Tablas y grficas con los datos (se requiere un nmero suficiente de muestras resultado de la experimentacin)

\chapter{Conclusiones y discusin}

Conclusiones de consecucin de objetivos planteados, a los experimentos
desarrollados y crtica final en base al anlisis de los datos.

\section{Cuestiones}
Como conclusin, qu cuestiones ms relevantes te plantea el trabajo? Al menos 3 cuestiones junto con su respuesta.


%% APENDICES
%\ \newpage
\appendix

\renewcommand{\baselinestretch}{1.2}

% BIBLIOGRAFIA
\bibliography{biblio}
\bibliographystyle{unsrt}
\cite{KM2000}
\end{document}
