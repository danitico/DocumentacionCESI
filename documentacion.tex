%%%%%%%%%%%%%%%%%%%%%%%%%%%%%%%%%%%%%%%%%%%%%%%%%%%%%%%%%
%                                                       %
% La clase: 11pt o 10pt, es un draft o copia definitiva %
% Que tipo de codificacion usa, ...                     %
%                                                       %
%%%%%%%%%%%%%%%%%%%%%%%%%%%%%%%%%%%%%%%%%%%%%%%%%%%%%%%%%
\documentclass[11pt,twoside,a4paper]{book}
\usepackage{fancyhdr}
\usepackage{fancybox}
\usepackage[T1]{fontenc}
\usepackage[utf8]{inputenc}
\usepackage[spanish]{babel}
\usepackage{amsfonts}
\usepackage{latexsym}
\usepackage{graphicx}
\usepackage{floatflt}
\usepackage{epsfig}
\usepackage{subfigure}
\usepackage{mathrsfs}
\usepackage{amssymb}


\renewcommand{\baselinestretch}{1.2}
\title{\Huge Servidor Web Tomcat}
\author{José Manuel Cuevas Muñoz, Daniel Ranchal Parrado, Carlos Romeo Muñoz}
\date{9 de Diciembre 2018}
%%%%%%%%%%%%%%%%%%%%%%%%%%%
%                         %
% Comenzamos el documento %
%                         %
%%%%%%%%%%%%%%%%%%%%%%%%%%%
\begin{document}
\maketitle
%%%%%%%%%%%%%%%%%%%%%%%%%%%
%
% Algunos ajustes previos
%
%%%%%%%%%%%%%%%%%%%%%%%%%%%

\renewcommand\bibname{Bibliografía}
\renewcommand\tablename{Tabla}

\pagenumbering{arabic} \fancyhf{} \pagestyle{fancy}
\fancyhead[LO]{\rightmark} % En las pginas impares, parte izquierda del encabezado, aparecer el nombre de seccin
\fancyhead[RE]{\leftmark} % En las pginas pares, parte derecha del encabezado, aparecer el nombre de captulo
\fancyhead[RO,LE]{\thepage} % Nmeros de pgina en las esquinas de los encabezados

%Aumentamos el tamao de la cabecera
\setlength{\headheight}{14pt}


\renewcommand{\sectionmark}[1]{\markright{{\thesection. #1}}} % Formato para la seccin: N.M. Nombre



%%%
%con esto creamos el ndice
\tableofcontents
%
% Contenido TRABAJO
%

\chapter{Definición del Sistema}
\section{Objetivos y definición del sistema}
El objetivo que se pretende es hacer pruebas exhaustivas, es decir, hacer un benchmarking para calcular el rendimiento del servicio web Tomcat. Para hacer este tipo de pruebas, se parte de la siguiente información.
El sistema operativo bajo el que se están haciendo las pruebas es CentOS, cuyo servidor está alojado en Azure. Para la instalación de Tomcat, se ha utilizado el servicio de virtualización docker para instalarlo de una manera sencilla. Lo único que hay que hacer es mapear el puerto del servidor web de nuestro contenedor al puerto 80 del host. En los archivos adjuntos y en la presentación se hará una breve explicación.

\section{Servicios y sus posibles resultados}
Un sistema puede dar un resultado vlido, invlido o simplemente no dar ningn resultado, en cualquier caso, habr que medir la tasa de sucesos de uno u otro tipo.

\section{Mtricas}
Fijar los criterios para comparar prestaciones. Qu magnitudes vamos a usar.


\section{Parmetros}
Listar los parmetros que puedan afectar a las prestaciones. Las caractersticas del sistema sern iguales a igualdad de hardware y software.


\chapter{Evaluacin del Sistema}
Definicin de la experimentacin a realizar. Qu evaluacin se seguir, cual ser la carga de trabajo, qu experimentos se van a realizar y finalmente se analizarn los resultados de la experimentacin.

\section{Tcnicas de evaluacin}
Qu tcnica de evaluacin se seguir.

\section{Carga de trabajo}
Cual ser la carga de trabajo, en qu consiste.

\section{Diseo de Experimentos}
Cmo se han diseado los experimentos. En qu consiste la experimentacin a realizar.

\section{Anlisis de los resultados}
Estudio y anlisis de los resultados provenientes de la experimentacin. Tablas y grficas con los datos (se requiere un nmero suficiente de muestras resultado de la experimentacin)

\chapter{Conclusiones y discusin}

Conclusiones de consecucin de objetivos planteados, a los experimentos
desarrollados y crtica final en base al anlisis de los datos.

\section{Cuestiones}
Como conclusin, qu cuestiones ms relevantes te plantea el trabajo? Al menos 3 cuestiones junto con su respuesta.


%% APENDICES
%\ \newpage
\appendix

\renewcommand{\baselinestretch}{1.2}

% BIBLIOGRAFIA
\bibliography{biblio}
\bibliographystyle{unsrt}
\cite{KM2000}
\end{document}
